% !TeX root = ../../infdesc.tex
\subsection*{Properties of relations}

\begin{chapex}
For each of the eight subsets
$$P \subseteq \{ \text{reflexive}, \text{symmetric}, \text{transitive} \}$$
find a relation satisfying (only) the properties in $P$.
\end{chapex}

\begin{chapex}
Prove that if $R$ is a symmetric, antisymmetric relation on a set $X$, then it is a subrelation of the equality relation---that is, $\mathrm{Gr}(R) \subseteq \mathrm{Gr}(=)$.
\end{chapex}

\begin{chapex}
A relation $R$ on a set $X$ is \textbf{left-total}\index{relation!left-total} if for all $x \in X$, there exists some $y \in X$ such that $x \mathbin{R} y$. Prove that every left-total, symmetric, transitive relation is reflexive.
\end{chapex}

% In \Crefrange{cqVerifyPropertiesOfRelationsBegin}{cqVerifyPropertiesOfRelationsEnd}, determine whether the relation is (a) reflexive, (b) symmetric, (c) antisymmetric, and (d) transitive.

% \begin{chapex}
% \label{cqVerifyPropertiesOfRelationsBegin}

% \end{chapex}

% \begin{chapex}

% \end{chapex}

% \begin{chapex}

% \end{chapex}

% \begin{chapex}

% \end{chapex}

% \begin{chapex}

% \end{chapex}

% \begin{chapex}
% \label{cqVerifyPropertiesOfRelationsEnd}

% \end{chapex}

\subsection*{Equivalence relations}

\begin{definition}
\label{defTransportOfRelation}
Let $R$ be a relation on a set $X$ and let $f : X \to Y$ be a function. The \textbf{transport} of $R$ along $f$ is the relation $S$ on $Y$ defined for $c,d \in Y$ by letting $c \mathrel{S} d$ if and only if there exist $a,b \in X$ such that $f(a)=c$, $f(b)=d$ and $a \mathrel{R} b$. That is
\[ \mathrm{Gr}(S) = \{ (f(a), f(b)) \mid a,b \in X,~ a \mathrel{R} b \} \]
\end{definition}

\begin{chapex}
Let $X$ and $Y$ be sets and let $f : X \to Y$. Prove that if $\sim$ is an equivalence relation on $X$, then the transport of $\sim$ along $f$ is an equivalence relation on $Y$.
\end{chapex}

\begin{definition}
\label{defEquivalenceRelationGeneratedByRelation}
\index{equivalence relation!generated by a relation}
\index{generated!equivalence relation}
\nindex{equivalence relation generated by $R$}{$\sim_R$}{equivalence relation generated by $R$}
Let $R$ be any relation on a set $X$. The \textbf{equivalence relation generated by $R$} is the relation ${\sim}_R$ on $X$ defined as follows. Given $x,y \in X$, say $x \sim_R y$ if and only if for some $k \in \mathbb{N}$ there is a sequence $(a_0, a_1, \dots, a_k)$ of elements of $X$ such that $a_0=x$, $a_k=y$ and, for all $0 \le i < k$, either $a_i\,R\,a_{i+1}$ or $a_{i+1}\,R\,a_i$.
\end{definition}

\begin{chapex}
\label{cqEquivalenceRelationGeneratedByRelationBegin}
Fix $n \in \mathbb{Z}$ and let $R$ be the relation on $\mathbb{Z}$ defined by $x\,R\,y$ if and only if $y=x+n$. Prove that ${\sim}_R$ is the relation of congruence modulo $n$.
\end{chapex}

\begin{chapex}
Let $X$ be a set and let $R$ be the subset relation on $\mathcal{P}(X)$. Prove that $U \mathrel{{\sim}_R} V$ for all $U, V \subseteq X$.
\end{chapex}

\begin{chapex}
\label{cqEquivalenceRelationGeneratedByRelationEnd}
Let $X$ be a set, fix two distinct elements $a,b \in X$, and define a relation $R$ on $X$ by declaring $a\,R\,b$ only---that is, for all $x,y \in X$, we have $x\,R\,y$ if and only if $x=a$ and $y=b$. Prove that the relation ${\sim}_R$ is defined by $x \sim_R y$ if and only if either $x=y$ or $\{x,y\} = \{a,b\}$.
\end{chapex}

In \Crefrange{cqPropertiesOfEquivalenceRelationGeneratedByRelationBegin}{cqPropertiesOfEquivalenceRelationGeneratedByRelationEnd}, let $R$ be a relation on a set $X$, and let $\sim_R$ be the equivalence relation generated by $R$ (as in \Cref{defEquivalenceRelationGeneratedByRelation}). In these questions, you will prove that $\sim_R$ is the `smallest' equivalence relation extending $R$.

\begin{chapex}
\label{cqPropertiesOfEquivalenceRelationGeneratedByRelationBegin}
Prove that ${\sim}_R$ is an equivalence relation on $X$.
\end{chapex}

\begin{chapex}
Prove that $x\,R\,y \Rightarrow x \sim_R y$ for all $x,y \in X$.
\end{chapex}

\begin{chapex}
Prove that if $\approx$ is any equivalence relation on $X$ and $x\,R\,y \Rightarrow x \approx y$ for all $x,y \in X$, then $x \sim_R y \Rightarrow x \approx y$ for all $x,y \in X$.
\end{chapex}

\begin{chapex}
\label{cqPropertiesOfEquivalenceRelationGeneratedByRelationEnd}
Prove that if $R$ is an equivalence relation, then ${\sim_R} = R$.
\end{chapex}

\begin{chapex}
\label{cqAntiderivativesIndefiniteIntegrals}
This question requires some familiarity with concepts from calculus. Let $\mathcal{C}^1(\mathbb{R})$ denote the set of all differentiable functions $F : \mathbb{R} \to \mathbb{R}$ whose derivative $F' : \mathbb{R} \to \mathbb{R}$ is continuous, and define a relation $\sim$ on $\mathcal{C}^1(\mathbb{R})$ by letting $F \sim G$ if and only if $F' = G'$.
\begin{enumerate}[(a)]
\item Prove that $\sim$ is an equivalence relation on $\mathcal{C}^1(\mathbb{R})$.
\item Let $F \in \mathcal{C}^1(\mathbb{R})$. Prove that
\[ [F]_{\sim} = \{ G \in \mathcal{C}^1(\mathbb{R}) \mid \exists c \in \mathbb{R},\, \forall x \in \mathbb{R},\, G(x) = F(x) + c \} \]
\item Interpret the notion of an indefinite integral $\displaystyle \int f(x)\, dx$ of a continuous function $f : \mathbb{R} \to \mathbb{R}$ in terms of $\sim$-equivalence classes.
\end{enumerate}
\end{chapex}

\subsection*{True--False questions}

\tfquestiontext{cqRelationsTFBegin}{cqRelationsTFEnd}

\begin{chapex} % False
\label{cqRelationsTFBegin}
Every reflexive relation is symmetric.
\end{chapex}

\begin{chapex} % True
There is a unique relation on $\mathbb{N}$ that is reflexive, symmetric and antisymmetric.
\end{chapex}

\begin{chapex} % False
Every relation that is not symmetric is antisymmetric.
\end{chapex}

\begin{chapex} % False
Every symmetric, antisymmetric relation is reflexive.
\end{chapex}

\begin{chapex} % True
The empty set is the graph of a relation on $\mathbb{N}$.
\end{chapex}

\begin{chapex} % True
For all sets $X$, the graph of a function $X \to X$ is the graph of a relation on $X$.
\end{chapex}

\begin{chapex} % False
\label{cqRelationsTFEnd}
For all sets $X$, the graph of a relation on $X$ is the graph of a function $X \to X$.
\end{chapex}

\subsection*{Always--Sometimes--Never questions}

\asnquestiontext{cqRelationsASNBegin}{cqRelationsASNEnd}

\begin{chapex} % Sometimes
\label{cqRelationsASNBegin}
Let $\sim$ be an equivalence relation on a set $X$. Then the relation $\approx$ on $\mathcal{P}(X) \setminus \{ \varnothing \}$, defined by $A \approx B$ if and only if $x \sim y$ for all $x \in A$ and $y \in B$, is an equivalence relation.
\end{chapex}

\begin{chapex} % Always
\label{cqRelationsASNEnd}
Let $\sim$ be an equivalence relation on a set $X$. Then the relation $\approx$ on $\mathcal{P}(X) \setminus \{ \varnothing \}$, defined by $A \approx B$ if and only if $x \sim y$ for some $x \in A$ and $y \in B$, is an equivalence relation.
\end{chapex}