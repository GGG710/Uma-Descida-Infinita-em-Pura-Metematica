% !TeX root = ../../book.tex
\chapter*{Acknowledgements}
\addcontentsline{toc}{chapter}{Acknowledgements}
\markboth{Acknowledgements}{Acknowledgements}

When I reflect on the time I have spent writing this book, I am overwhelmed by the number of people who have had some kind of influence on its content.

This book would never have come to exist were it not for Chad Hershock's course 38-801 \textit{Evidence-Based Teaching in the Sciences}, which I took in Fall 2014 as a graduate student at Carnegie Mellon University. His course heavily influenced my approach to teaching, and it motivated me to write this book in the first place. Many of the pedagogical decisions I made when writing this book were informed by research that I was exposed to as a student in Chad's class.

The legendary Carnegie Mellon professor, John Mackey, has been using this book (in various forms) as course notes for 21-128 \textit{Mathematical Concepts and Proofs} and 15-151 \textit{Mathematical Foundations of Computer Science} since Fall 2016. His influence can be felt throughout the book: thanks to discussions with John, many proofs have been reworded, sections restructured, and explanations improved. As a result, there is some overlap between the exercises in this book and the questions on his problem sheets. I am extremely grateful for his ongoing support.

Steve Awodey, who was my doctoral thesis advisor, has for a long time been a source of inspiration for me. Many of the choices I made when choosing how to present the material in this book are grounded in my desire to do mathematics \textit{the right way}---it was this desire that led me to study category theory, and ultimately to become Steve's PhD student. I learnt a great deal from him and I greatly appreciated his patience and flexibility in helping direct my research despite my busy teaching schedule and extracurricular interests (such as writing this book).

Perhaps unbeknownst to them, many insightful conversations with the following people have helped shape the material in this book in one way or another: Jeremy Avigad, Deb Brandon, Santiago Ca\~{n}ez, Heather Dwyer, Thomas Forster, Will Gunther, Kate Hamilton, Jessica Harrell, Bob Harper, Brian Kell, Marsha Lovett, Ben Millwood, David Offner, Ruth Poproski, Emily Riehl, Hilary Schuldt, Gareth Taylor, Katie Walsh, Emily Weiss and Andy Zucker.

The \textit{Stack Exchange} network has influenced the development of this book in two important ways. First, I have been an active member of \textit{Mathematics Stack Exchange} (\url{https://math.stackexchange.com/}) since early 2012 and have learnt a great deal about how to effectively explain mathematical concepts; occasionally, a question on Mathematics Stack Exchange inspires me to add a new example or exercise to the book. Second, I have made frequent use of \textit{\LaTeX{} Stack Exchange} (\url{https://tex.stackexchange.com}) for implementing some of the more technical aspects of writing a book using \LaTeX{}.

The Department of Mathematical Sciences at Carnegie Mellon University supported me academically, professionally and financially throughout my PhD and presented me with more opportunities than I could possibly have hoped for to develop as a teacher. This support is now continued by the Department of Mathematics at Northwestern University, where I am currently employed as a lecturer.

I would also like to thank everyone at Carnegie Mellon's and Northwestern's teaching centres, the Eberly Center and the Searle Center, respectively. Through various workshops, programs and fellowships at both teaching centres, I have learnt an incredible amount about how people learn, and I have transformed as a teacher. Their student-centred, evidence-based approach to the science of teaching and learning underlies everything I do as a teacher, including writing this book---their influence cannot be understated.

Finally, and importantly, I am grateful to the 1000+ students who have already used this book to learn mathematics. Every time a student contacts me to ask a question or point out an error, the book gets better; this is reflected in the dozens of typographical errors that have been fixed as a consequence.

\begin{flushright}
Clive Newstead\\
January 2020\\
Evanston, Illinois
\end{flushright}