% Options for packages loaded elsewhere
\PassOptionsToPackage{unicode}{hyperref}
\PassOptionsToPackage{hyphens}{url}
%
\documentclass[
]{article}
\usepackage{amsmath,amssymb}
\usepackage{lmodern}
\usepackage{iftex}
\ifPDFTeX
  \usepackage[T1]{fontenc}
  \usepackage[utf8]{inputenc}
  \usepackage{textcomp} % provide euro and other symbols
\else % if luatex or xetex
  \usepackage{unicode-math}
  \defaultfontfeatures{Scale=MatchLowercase}
  \defaultfontfeatures[\rmfamily]{Ligatures=TeX,Scale=1}
\fi
% Use upquote if available, for straight quotes in verbatim environments
\IfFileExists{upquote.sty}{\usepackage{upquote}}{}
\IfFileExists{microtype.sty}{% use microtype if available
  \usepackage[]{microtype}
  \UseMicrotypeSet[protrusion]{basicmath} % disable protrusion for tt fonts
}{}
\makeatletter
\@ifundefined{KOMAClassName}{% if non-KOMA class
  \IfFileExists{parskip.sty}{%
    \usepackage{parskip}
  }{% else
    \setlength{\parindent}{0pt}
    \setlength{\parskip}{6pt plus 2pt minus 1pt}}
}{% if KOMA class
  \KOMAoptions{parskip=half}}
\makeatother
\usepackage{xcolor}
\usepackage{longtable,booktabs,array}
\usepackage{multirow}
\usepackage{calc} % for calculating minipage widths
% Correct order of tables after \paragraph or \subparagraph
\usepackage{etoolbox}
\makeatletter
\patchcmd\longtable{\par}{\if@noskipsec\mbox{}\fi\par}{}{}
\makeatother
% Allow footnotes in longtable head/foot
\IfFileExists{footnotehyper.sty}{\usepackage{footnotehyper}}{\usepackage{footnote}}
\makesavenoteenv{longtable}
\usepackage{graphicx}
\makeatletter
\def\maxwidth{\ifdim\Gin@nat@width>\linewidth\linewidth\else\Gin@nat@width\fi}
\def\maxheight{\ifdim\Gin@nat@height>\textheight\textheight\else\Gin@nat@height\fi}
\makeatother
% Scale images if necessary, so that they will not overflow the page
% margins by default, and it is still possible to overwrite the defaults
% using explicit options in \includegraphics[width, height, ...]{}
\setkeys{Gin}{width=\maxwidth,height=\maxheight,keepaspectratio}
% Set default figure placement to htbp
\makeatletter
\def\fps@figure{htbp}
\makeatother
\setlength{\emergencystretch}{3em} % prevent overfull lines
\providecommand{\tightlist}{%
  \setlength{\itemsep}{0pt}\setlength{\parskip}{0pt}}
\setcounter{secnumdepth}{-\maxdimen} % remove section numbering
\ifLuaTeX
  \usepackage{selnolig}  % disable illegal ligatures
\fi
\IfFileExists{bookmark.sty}{\usepackage{bookmark}}{\usepackage{hyperref}}
\IfFileExists{xurl.sty}{\usepackage{xurl}}{} % add URL line breaks if available
\urlstyle{same} % disable monospaced font for URLs
\hypersetup{
  hidelinks,
  pdfcreator={LaTeX via pandoc}}

\author{}
\date{}

\begin{document}

\begin{quote}
Uma descida infinita em Matemática Pura
\end{quote}

\includegraphics[width=0.70694in,height=1.65972in]{vertopal_8f28c058ced248c09cffcbfdc362d1d6/media/image1.png}

POR CLIVE NEWSTEAD

\begin{quote}
\emph{Com adaptações de Jonas Galvao Xavier/Tradução por George Galvao}
\end{quote}

\emph{Moreira}

\emph{Última atualização em Monday 4th March 2024}

\begin{quote}
Trabalho Original © 2023 Clive Newstead, Todos Direitos Reservados.
\end{quote}

Adaptatações © 2024 Jonas Galvao Xavier, Todos Direitos Reservados.

Prévia da Primeira Edição(Em breve)

\begin{quote}
ISBN 978-1-950215-00-3 (Brochura) ISBN 978-1-950215-01-0 (Capa dura)

Uma cópia gratuita de \emph{Uma Descida Infinita Em Matemática Pura}
pode ser obtido através do site do livro:

Este livro, suas figuras e fontes TEX são lançadas sob uma atribuição
criativa internacional "Creative Commons Attribution--ShareAlike 4.0
International Licence". O texto completo da licença está replicado no
fim do livro, e pode ser achada no site Creative Commons:\\
\strut \\
\emph{Para meus Pais,}\\
\emph{Imogen e Matthew (1952--2021),} \emph{Que me ensinaram tanto.}
\end{quote}

iv

\begin{quote}
\textbf{Conteúdo}

Preface vii

Reconhecimentos xiii

0Começando 1
\end{quote}

\begin{longtable}[]{@{}
  >{\raggedright\arraybackslash}p{(\columnwidth - 6\tabcolsep) * \real{0.2500}}
  >{\raggedright\arraybackslash}p{(\columnwidth - 6\tabcolsep) * \real{0.2500}}
  >{\raggedright\arraybackslash}p{(\columnwidth - 6\tabcolsep) * \real{0.2500}}
  >{\raggedright\arraybackslash}p{(\columnwidth - 6\tabcolsep) * \real{0.2500}}@{}}
\toprule()
\multirow{2}{*}{\begin{minipage}[b]{\linewidth}\raggedright
I
\end{minipage}} & \begin{minipage}[b]{\linewidth}\raggedright
\begin{quote}
0.E
\end{quote}
\end{minipage} & \begin{minipage}[b]{\linewidth}\raggedright
Chapter 0 exercises . . . . . . . . . . . . . . . . . . . . . . . . . .
. .
\end{minipage} & \begin{minipage}[b]{\linewidth}\raggedright
20
\end{minipage} \\
&
\multicolumn{2}{>{\raggedright\arraybackslash}p{(\columnwidth - 6\tabcolsep) * \real{0.5000} + 2\tabcolsep}}{%
\begin{minipage}[b]{\linewidth}\raggedright
\begin{quote}
Conceitos centrais
\end{quote}
\end{minipage}} & \begin{minipage}[b]{\linewidth}\raggedright
23
\end{minipage} \\
\midrule()
\endhead
\bottomrule()
\end{longtable}

\emph{Contents}

\begin{longtable}[]{@{}
  >{\raggedright\arraybackslash}p{(\columnwidth - 6\tabcolsep) * \real{0.2500}}
  >{\raggedright\arraybackslash}p{(\columnwidth - 6\tabcolsep) * \real{0.2500}}
  >{\raggedright\arraybackslash}p{(\columnwidth - 6\tabcolsep) * \real{0.2500}}
  >{\raggedright\arraybackslash}p{(\columnwidth - 6\tabcolsep) * \real{0.2500}}@{}}
\toprule()
\multicolumn{2}{@{}>{\raggedright\arraybackslash}p{(\columnwidth - 6\tabcolsep) * \real{0.5000} + 2\tabcolsep}}{%
\begin{minipage}[b]{\linewidth}\raggedright
\begin{quote}
Apêndice
\end{quote}
\end{minipage}} &
\multirow{3}{*}{\begin{minipage}[b]{\linewidth}\raggedright
\begin{quote}
28
\end{quote}
\end{minipage}} &
\multirow{4}{*}{\begin{minipage}[b]{\linewidth}\raggedright
27
\end{minipage}} \\
\begin{minipage}[b]{\linewidth}\raggedright
\begin{quote}
AMiscelânea matemática
\end{quote}
\end{minipage} & \begin{minipage}[b]{\linewidth}\raggedright
\begin{quote}
27
\end{quote}
\end{minipage} \\
\multicolumn{2}{@{}>{\raggedright\arraybackslash}p{(\columnwidth - 6\tabcolsep) * \real{0.5000} + 2\tabcolsep}}{%
\begin{minipage}[b]{\linewidth}\raggedright
\begin{quote}
A.1Definir fundamentos teóricos . . . . . . . . . . . . . . . . . . . .
. . .
\end{quote}
\end{minipage}} \\
\multicolumn{2}{@{}>{\raggedright\arraybackslash}p{(\columnwidth - 6\tabcolsep) * \real{0.5000} + 2\tabcolsep}}{%
\begin{minipage}[b]{\linewidth}\raggedright
A.2Construções dos conjuntos de números . . . . . . . . . . . . . . . .
. .
\end{minipage}} & \begin{minipage}[b]{\linewidth}\raggedright
\begin{quote}
29
\end{quote}
\end{minipage} \\
\midrule()
\endhead
\bottomrule()
\end{longtable}

\begin{quote}
A.3Limites de funções . . . . . . . . . . . . . . . . . . . . . . . . .
. . . 47
\end{quote}

\begin{longtable}[]{@{}
  >{\raggedright\arraybackslash}p{(\columnwidth - 4\tabcolsep) * \real{0.3333}}
  >{\raggedright\arraybackslash}p{(\columnwidth - 4\tabcolsep) * \real{0.3333}}
  >{\raggedright\arraybackslash}p{(\columnwidth - 4\tabcolsep) * \real{0.3333}}@{}}
\toprule()
\begin{minipage}[b]{\linewidth}\raggedright
BDicas para exercícios selecionados
\end{minipage} &
\multirow{8}{*}{\begin{minipage}[b]{\linewidth}\raggedright
\begin{quote}
51
\end{quote}
\end{minipage}} &
\multirow{2}{*}{\begin{minipage}[b]{\linewidth}\raggedright
55
\end{minipage}} \\
\begin{minipage}[b]{\linewidth}\raggedright
\begin{quote}
Indices
\end{quote}
\end{minipage} \\
\begin{minipage}[b]{\linewidth}\raggedright
\begin{quote}
Índice de tópicos
\end{quote}
\end{minipage} & & \begin{minipage}[b]{\linewidth}\raggedright
55
\end{minipage} \\
\begin{minipage}[b]{\linewidth}\raggedright
\begin{quote}
Índice de vocabulário
\end{quote}
\end{minipage} & & \begin{minipage}[b]{\linewidth}\raggedright
55
\end{minipage} \\
\begin{minipage}[b]{\linewidth}\raggedright
\begin{quote}
Índice de notação
\end{quote}
\end{minipage} & & \begin{minipage}[b]{\linewidth}\raggedright
57
\end{minipage} \\
\multirow{2}{*}{\begin{minipage}[b]{\linewidth}\raggedright
\begin{quote}
Índices de comandos LATEX
\end{quote}
\end{minipage}} & & \begin{minipage}[b]{\linewidth}\raggedright
57
\end{minipage} \\
& & \multirow{2}{*}{\begin{minipage}[b]{\linewidth}\raggedright
61
\end{minipage}} \\
\begin{minipage}[b]{\linewidth}\raggedright
\begin{quote}
Licence
\end{quote}
\end{minipage} \\
\midrule()
\endhead
\bottomrule()
\end{longtable}

iv

\textbf{Prefácio}

\begin{quote}
Olá, e obrigado por tirar tempo pra ler essa rápida introdução à
\emph{Uma Descida Infinita em Pura Matemática}! A versão mais recente do
livro, em inglês, está gratuitamente disponível para download no
seguinte site:
\end{quote}

O site também inclui informações sobre mudanças entre diferentes versões
do livro,

\begin{quote}
um arquivo de versões prévias, e alguns recursos para usar LATEX (veja
também ??).
\end{quote}

Sobre o livro

\begin{quote}
Um estudante em uma típica classe de cálculo vai aprender sobre a "regra
da cadeia" e posteriormente a como usa-lá para resolver alguns problemas
pré-escritos de "regra da cadeia" assim como calcular a derivada de
sin(1+\emph{x}2) com respeito a \emph{x}, ou talvez resolver um problema
de palavras envolvendo taxas de variação relacionadas a mudança. A
expectativa é que o estudante aplique corretamente a regra da cadeia
para derivar a resposta correta, e mostrar trabalho suficiente para ser
crível. Nesse sentido, o estudante é um \emph{consumidor} da matemática.
Eles recebem a regra da cadeia como uma ferramenta, a ser aceita sem
questionamento, para então resolver um grupo pequeno de problemas.

O objetivo desse livro é ajudar o leitor a fazer a transição: de um
\emph{consumidor} da matemática para um \emph{produtor} dela. É isso que
significa `pura' matemática. Enquanto um consumidor da matemática pode
aprender a regra da cadeia e usa-la para calcular uma derivada, um
produtor da matemática pode derivar a regra da cadeia a partir de uma
rigorosa definição de uma derivada, e então provar mais versões
abstratas da regra da cadeia em contextos mais gerais (como análise
multivariável).

Dos consumidores da matemática é esperado que digam como usaram suas
ferramentas para encontrar suas respostas. Produtores da matemática, por
outro lado, tem muito mais a fazer: Eles precisam estar prontos para
acompanhar as definições e hipóteses,
\end{quote}

v

vi \emph{Prefácio} juntar os dados em novas e interessantes formas, e
fazer suas próprias definições de conceitos matemáticos. Ainda mais, uma
vez que eles fizeram isso, eles devem comunicar suas descobertas de uma
forma que outros considerem inteligível, e eles precisam convencer
outros que aquilo que eles fizeram está correto, apropriado e que vale a
pena.

É essa a transição do consumo para produção de matemática que guiou os
princípios usados para planejar e escrever esse livro. Em particular:

•Comunicação. Acima de tudo, este livro visa ajudar o leitor a obter
alfabetização matemática e se expressar matematicamente. Isso ocorre em
muitos níveis de ampliação. Por exemplo, considere a seguinte expressão:

∀\emph{x} ∈ R\emph{,} {[}¬(\emph{x} = 0) ⇒ (∃\emph{y} ∈ R\emph{, y}2
\emph{\textless{} x}2){]}

\begin{quote}
Depois de trabalhar com esse livro, você será capaz de dizer o que os
símbolos ∀, ∈, R, ¬, ⇒ e ∃ significam intuitivamente e como eles são
definidos precisamente. Mas

você terá também que saber interpretar o que a expressão significa como
um todo,

explicar o que significa em termos simples para que outra pessoa entenda
\emph{sem} usar

símbolos confusos, provar que aquilo é verdade, e comunicar sua prova
para outra

pessoa de uma forma clara e concisa.

Os tipos de ferramentas necessárias para fazer isso estão desenvolvidas
nos capítulos principais do livro, e mais foco é dado para o lado
escrito das coisas.
\end{quote}

•Inquérito. As pessoas aprendem mais quando elas descobrem as coisas por
elas mesmas. Usando esse princípio ao extremo o livro estaria em branco.
No entanto, eu acredito que é importante incorporar aspectos da
aprendizagem baseada em investigação no texto.

\begin{quote}
Este princípio manifesta-se na medida que existem exercícios espalhados
pelo texto, muitos dos quais simplesmente requerem que você prove um
resultado. Muitos leitores vão achar isso frustrante, mas isso é por um
bom motivo: esses exercícios servem como pontos de controle para
garantir que seu entendimento desse material é suficiente para proceder.
Aquele sentimento de frustração é o que se chamaria de aprendizado ---
abrace-o!
\end{quote}

•Estratégia. Uma prova matemática é muito parecida com um quebra-cabeça.
Em qualquer fase dada em uma prova, você vai ter algumas definições,
premissas e resultados que estão disponíveis para ser usados, e você
precisa junta-los usando as regras lógicas à sua disposição. Por todo
livro, e particularmente nos primeiros capítulos, eu fiz um esforço para
destacar estratégias úteis de prova em qualquer lugar que elas surjam.

vi

•Conteúdo. Não há muito sentido em aprender matemática se você não tem
nenhum conceito para prova. Com isso em mente, ?? inclui vários
capítulos dedicados à introdução de algumas áreas temáticas em
matemática pura, tanto quanto teoria dos números, combinatórias, análise
e teoria da probabilidade.

\begin{quote}
\emph{Prefácio} ix
\end{quote}

•LATEX. O \emph{de facto} padrão para composição matemática é o LATEX.
Eu acho que é importante para matemáticos aprender sobre ele cedo de uma
forma guiada, então

\begin{quote}
eu escrevi um breve tutorial no ?? e inclui código LATEX para toda nova
notação conforme definida ao longo do livro.

Navegando o livro

Esse livro não precisa, e enfaticamente \emph{não deve}, ser lido de
frente para trás. A ordem desse material foi escolhida de forma que o
material que aparece depois depende apenas do material que aparece
antes, mas seguir o material na ordem que é apresentado pode ser uma
experiência bastante seca.

A maioria dos cursos introdutórios de matemática pura abrange, no
mínimo, lógica simbólica, conjuntos, funções e relações. Este material é
o conteúdo do Part I. Tais cursos frequentemente abrangem tópicos
adicionais da matemática pura, com exatamente \emph{quais} tópicos
dependendo do que o curso está preparando os estudantes. Com isso em
mente, ?? serve como uma introdução para uma gama de áreas de matemática
pura, incluindo teoria dos números, combinatórias, teoria dos conjuntos,
análise real, teoria das probabilidades e teoria da ordem.

Não é necessário cobrir toda a Part I antes de seguir os tópicos. Na
verdade, intercalar material pode ser uma maneira útil de motivar muitos
dos conceitos abstratos que surgem na Part I.

A seguinte tabela mostra dependências entre seções. Seções prévias
dentro do mesmo capítulo que uma seção deveria ser considerado
`essencial' pré-requisitos, a menos que indicado de outra forma.
\end{quote}

\begin{longtable}[]{@{}
  >{\raggedright\arraybackslash}p{(\columnwidth - 4\tabcolsep) * \real{0.3333}}
  >{\raggedright\arraybackslash}p{(\columnwidth - 4\tabcolsep) * \real{0.3333}}
  >{\raggedright\arraybackslash}p{(\columnwidth - 4\tabcolsep) * \real{0.3333}}@{}}
\toprule()
\begin{minipage}[b]{\linewidth}\raggedright
Parte
\end{minipage} & \begin{minipage}[b]{\linewidth}\raggedright
Seção
\end{minipage} & \begin{minipage}[b]{\linewidth}\raggedright
\begin{longtable}[]{@{}
  >{\raggedright\arraybackslash}p{(\columnwidth - 4\tabcolsep) * \real{0.3333}}
  >{\raggedright\arraybackslash}p{(\columnwidth - 4\tabcolsep) * \real{0.3333}}
  >{\raggedright\arraybackslash}p{(\columnwidth - 4\tabcolsep) * \real{0.3333}}@{}}
\toprule()
\begin{minipage}[b]{\linewidth}\raggedright
Essencial
\end{minipage} & \begin{minipage}[b]{\linewidth}\raggedright
Recomendado
\end{minipage} & \begin{minipage}[b]{\linewidth}\raggedright
útil
\end{minipage} \\
\midrule()
\endhead
\bottomrule()
\end{longtable}
\end{minipage} \\
\midrule()
\endhead
\bottomrule()
\end{longtable}

\begin{longtable}[]{@{}
  >{\raggedright\arraybackslash}p{(\columnwidth - 4\tabcolsep) * \real{0.3333}}
  >{\raggedright\arraybackslash}p{(\columnwidth - 4\tabcolsep) * \real{0.3333}}
  >{\raggedright\arraybackslash}p{(\columnwidth - 4\tabcolsep) * \real{0.3333}}@{}}
\toprule()
\multirow{2}{*}{\begin{minipage}[b]{\linewidth}\raggedright
I
\end{minipage}} & \begin{minipage}[b]{\linewidth}\raggedright
??
\end{minipage} & \begin{minipage}[b]{\linewidth}\raggedright
\begin{quote}
0
\end{quote}
\end{minipage} \\
& \begin{minipage}[b]{\linewidth}\raggedright
??
\end{minipage} & \begin{minipage}[b]{\linewidth}\raggedright
\begin{quote}
??
\end{quote}
\end{minipage} \\
\midrule()
\endhead
\bottomrule()
\end{longtable}

vii

\begin{longtable}[]{@{}
  >{\raggedright\arraybackslash}p{(\columnwidth - 10\tabcolsep) * \real{0.1667}}
  >{\raggedright\arraybackslash}p{(\columnwidth - 10\tabcolsep) * \real{0.1667}}
  >{\raggedright\arraybackslash}p{(\columnwidth - 10\tabcolsep) * \real{0.1667}}
  >{\raggedright\arraybackslash}p{(\columnwidth - 10\tabcolsep) * \real{0.1667}}
  >{\raggedright\arraybackslash}p{(\columnwidth - 10\tabcolsep) * \real{0.1667}}
  >{\raggedright\arraybackslash}p{(\columnwidth - 10\tabcolsep) * \real{0.1667}}@{}}
\toprule()
\multirow{3}{*}{\begin{minipage}[b]{\linewidth}\raggedright
\begin{quote}
viii
\end{quote}
\end{minipage}} & \begin{minipage}[b]{\linewidth}\raggedright
??
\end{minipage} & \begin{minipage}[b]{\linewidth}\raggedright
??
\end{minipage} &
\multirow{2}{*}{\begin{minipage}[b]{\linewidth}\raggedright
??
\end{minipage}} &
\multirow{2}{*}{\begin{minipage}[b]{\linewidth}\raggedright
??
\end{minipage}} &
\multirow{3}{*}{\begin{minipage}[b]{\linewidth}\raggedright
\emph{Prefácio}
\end{minipage}} \\
& \begin{minipage}[b]{\linewidth}\raggedright
??
\end{minipage} & \begin{minipage}[b]{\linewidth}\raggedright
??
\end{minipage} \\
& \begin{minipage}[b]{\linewidth}\raggedright
??
\end{minipage} & \begin{minipage}[b]{\linewidth}\raggedright
??
\end{minipage} & \begin{minipage}[b]{\linewidth}\raggedright
??
\end{minipage} & \begin{minipage}[b]{\linewidth}\raggedright
??, ??
\end{minipage} \\
\midrule()
\endhead
\bottomrule()
\end{longtable}

\begin{longtable}[]{@{}
  >{\raggedright\arraybackslash}p{(\columnwidth - 4\tabcolsep) * \real{0.3333}}
  >{\raggedright\arraybackslash}p{(\columnwidth - 4\tabcolsep) * \real{0.3333}}
  >{\raggedright\arraybackslash}p{(\columnwidth - 4\tabcolsep) * \real{0.3333}}@{}}
\toprule()
\begin{minipage}[b]{\linewidth}\raggedright
\end{minipage} & \begin{minipage}[b]{\linewidth}\raggedright
??
\end{minipage} & \begin{minipage}[b]{\linewidth}\raggedright
\begin{quote}
??, ?? ??
\end{quote}
\end{minipage} \\
\midrule()
\endhead
\bottomrule()
\end{longtable}

\begin{longtable}[]{@{}
  >{\raggedright\arraybackslash}p{(\columnwidth - 8\tabcolsep) * \real{0.2000}}
  >{\raggedright\arraybackslash}p{(\columnwidth - 8\tabcolsep) * \real{0.2000}}
  >{\raggedright\arraybackslash}p{(\columnwidth - 8\tabcolsep) * \real{0.2000}}
  >{\raggedright\arraybackslash}p{(\columnwidth - 8\tabcolsep) * \real{0.2000}}
  >{\raggedright\arraybackslash}p{(\columnwidth - 8\tabcolsep) * \real{0.2000}}@{}}
\toprule()
\multirow{12}{*}{\begin{minipage}[b]{\linewidth}\raggedright
??
\end{minipage}} & \begin{minipage}[b]{\linewidth}\raggedright
??
\end{minipage} & \begin{minipage}[b]{\linewidth}\raggedright
??
\end{minipage} & \begin{minipage}[b]{\linewidth}\raggedright
??, ??
\end{minipage} & \begin{minipage}[b]{\linewidth}\raggedright
??
\end{minipage} \\
& \begin{minipage}[b]{\linewidth}\raggedright
??
\end{minipage} &
\multirow{2}{*}{\begin{minipage}[b]{\linewidth}\raggedright
??
\end{minipage}} &
\multirow{3}{*}{\begin{minipage}[b]{\linewidth}\raggedright
??
\end{minipage}} &
\multirow{3}{*}{\begin{minipage}[b]{\linewidth}\raggedright
??
\end{minipage}} \\
& \begin{minipage}[b]{\linewidth}\raggedright
??
\end{minipage} \\
& \begin{minipage}[b]{\linewidth}\raggedright
??
\end{minipage} & \begin{minipage}[b]{\linewidth}\raggedright
\begin{quote}
??, ??
\end{quote}
\end{minipage} \\
& \begin{minipage}[b]{\linewidth}\raggedright
??
\end{minipage} & \begin{minipage}[b]{\linewidth}\raggedright
??
\end{minipage} & \begin{minipage}[b]{\linewidth}\raggedright
??
\end{minipage} &
\multirow{2}{*}{\begin{minipage}[b]{\linewidth}\raggedright
\begin{quote}
??, ??
\end{quote}
\end{minipage}} \\
& \begin{minipage}[b]{\linewidth}\raggedright
??
\end{minipage} & \begin{minipage}[b]{\linewidth}\raggedright
??
\end{minipage} & \begin{minipage}[b]{\linewidth}\raggedright
??
\end{minipage} \\
& \begin{minipage}[b]{\linewidth}\raggedright
??
\end{minipage} & \begin{minipage}[b]{\linewidth}\raggedright
??
\end{minipage} &
\multirow{4}{*}{\begin{minipage}[b]{\linewidth}\raggedright
??, ??
\end{minipage}} &
\multirow{6}{*}{\begin{minipage}[b]{\linewidth}\raggedright
??
\end{minipage}} \\
& \begin{minipage}[b]{\linewidth}\raggedright
??
\end{minipage} & \begin{minipage}[b]{\linewidth}\raggedright
??
\end{minipage} \\
& \begin{minipage}[b]{\linewidth}\raggedright
??
\end{minipage} & \begin{minipage}[b]{\linewidth}\raggedright
??
\end{minipage} \\
& \multirow{2}{*}{\begin{minipage}[b]{\linewidth}\raggedright
??
\end{minipage}} &
\multirow{2}{*}{\begin{minipage}[b]{\linewidth}\raggedright
??
\end{minipage}} \\
& & & \multirow{2}{*}{\begin{minipage}[b]{\linewidth}\raggedright
??
\end{minipage}} \\
& \begin{minipage}[b]{\linewidth}\raggedright
??
\end{minipage} & \begin{minipage}[b]{\linewidth}\raggedright
\begin{quote}
??, ??
\end{quote}
\end{minipage} \\
\midrule()
\endhead
\bottomrule()
\end{longtable}

\begin{quote}
Pré-requisitos são cumulativos. Por exemplo, a fim de cobrir ??, você
deveria primeiro cobrir 0?.
\end{quote}

O que os números, cores e símbolos significam

\begin{quote}
Falando amplamente, o material nesse livro está fragmentado em itens
enumerados que em uma de cinco categorias: definições, resultados,
observações, exemplos e exercícios. Na ??, nós também temos extratos de
provas. Para melhorar navegabilidade, essas categorias são distintas por
nome, cor e símbolo, como indicado na seguinte tabela.
\end{quote}

\begin{longtable}[]{@{}
  >{\raggedright\arraybackslash}p{(\columnwidth - 4\tabcolsep) * \real{0.3333}}
  >{\raggedright\arraybackslash}p{(\columnwidth - 4\tabcolsep) * \real{0.3333}}
  >{\raggedright\arraybackslash}p{(\columnwidth - 4\tabcolsep) * \real{0.3333}}@{}}
\toprule()
\begin{minipage}[b]{\linewidth}\raggedright
\begin{quote}
Categoria
\end{quote}
\end{minipage} & \begin{minipage}[b]{\linewidth}\raggedright
\begin{quote}
Símbolo
\end{quote}
\end{minipage} & \begin{minipage}[b]{\linewidth}\raggedright
\begin{quote}
Cor
\end{quote}
\end{minipage} \\
\midrule()
\endhead
Definições & ✦ & \textbf{Vermelho} \\
\bottomrule()
\end{longtable}

viii

\begin{longtable}[]{@{}
  >{\raggedright\arraybackslash}p{(\columnwidth - 6\tabcolsep) * \real{0.2500}}
  >{\raggedright\arraybackslash}p{(\columnwidth - 6\tabcolsep) * \real{0.2500}}
  >{\raggedright\arraybackslash}p{(\columnwidth - 6\tabcolsep) * \real{0.2500}}
  >{\raggedright\arraybackslash}p{(\columnwidth - 6\tabcolsep) * \real{0.2500}}@{}}
\toprule()
\multirow{3}{*}{\begin{minipage}[b]{\linewidth}\raggedright
\begin{quote}
Resultados\\
Observações Categoria
\end{quote}\strut
\end{minipage}} & \begin{minipage}[b]{\linewidth}\raggedright
\begin{quote}
✣
\end{quote}
\end{minipage} &
\multicolumn{2}{>{\raggedright\arraybackslash}p{(\columnwidth - 6\tabcolsep) * \real{0.5000} + 2\tabcolsep}@{}}{%
\begin{minipage}[b]{\linewidth}\raggedright
\begin{quote}
\textbf{Azul}
\end{quote}
\end{minipage}} \\
& \begin{minipage}[b]{\linewidth}\raggedright
\begin{quote}
❖
\end{quote}
\end{minipage} &
\multicolumn{2}{>{\raggedright\arraybackslash}p{(\columnwidth - 6\tabcolsep) * \real{0.5000} + 2\tabcolsep}@{}}{%
\begin{minipage}[b]{\linewidth}\raggedright
\begin{quote}
\textbf{Roxo}
\end{quote}
\end{minipage}} \\
&
\multicolumn{2}{>{\raggedright\arraybackslash}p{(\columnwidth - 6\tabcolsep) * \real{0.5000} + 2\tabcolsep}}{%
\begin{minipage}[b]{\linewidth}\raggedright
Símbolo
\end{minipage}} & \begin{minipage}[b]{\linewidth}\raggedright
\begin{quote}
Cor
\end{quote}
\end{minipage} \\
\midrule()
\endhead
\multirow{3}{*}{\begin{minipage}[t]{\linewidth}\raggedright
\begin{quote}
Exemplos\\
Exercícios\\
Extratos de provas
\end{quote}\strut
\end{minipage}} & ✐ &
\multicolumn{2}{>{\raggedright\arraybackslash}p{(\columnwidth - 6\tabcolsep) * \real{0.5000} + 2\tabcolsep}@{}}{%
\begin{minipage}[t]{\linewidth}\raggedright
\begin{quote}
\textbf{Verde}
\end{quote}
\end{minipage}} \\
& ✎ &
\multicolumn{2}{>{\raggedright\arraybackslash}p{(\columnwidth - 6\tabcolsep) * \real{0.5000} + 2\tabcolsep}@{}}{%
\textbf{Dourado}} \\
& ❝ &
\multicolumn{2}{>{\raggedright\arraybackslash}p{(\columnwidth - 6\tabcolsep) * \real{0.5000} + 2\tabcolsep}@{}}{%
\begin{minipage}[t]{\linewidth}\raggedright
\begin{quote}
\textbf{Verde}
\end{quote}
\end{minipage}} \\
\bottomrule()
\end{longtable}

Estes itens estão enumerados de acordo com próprias seções. Definições e
teoremas

\begin{longtable}[]{@{}
  >{\raggedright\arraybackslash}p{(\columnwidth - 2\tabcolsep) * \real{0.5000}}
  >{\raggedright\arraybackslash}p{(\columnwidth - 2\tabcolsep) * \real{0.5000}}@{}}
\toprule()
\begin{minipage}[b]{\linewidth}\raggedright
\begin{quote}
(Resultados importantes) aparecem em uma caixa .
\end{quote}

\begin{longtable}[]{@{}
  >{\raggedright\arraybackslash}p{(\columnwidth - 0\tabcolsep) * \real{1.0000}}@{}}
\toprule()
\begin{minipage}[b]{\linewidth}\raggedright
\end{minipage} \\
\midrule()
\endhead
\bottomrule()
\end{longtable}

\begin{quote}
\emph{Prefácio}
\end{quote}
\end{minipage} & \begin{minipage}[b]{\linewidth}\raggedright
xi
\end{minipage} \\
\midrule()
\endhead
\bottomrule()
\end{longtable}

Você também vai encontrar os símbolos □ e ◁ cujos significados são os
seguintes:

\begin{quote}
□ Fim de uma prova. É um padrão em documentos matemáticos para
identificar quando uma prova acabou por desenhar um pequeno quadrado ou
por escrever `\emph{Q.E.D.}' (Este último significa \emph{quod erat
demonstrandum}, que é latim para \emph{que deveria ser mostrado}.)

◁ Fim do item. Isso \emph{não é} um uso padrão, e está incluído apenas
para ajudá-lo a identificar quando um item foi concluído e o conteúdo
principal do livro continua.

Algumas subjeções são rotuladas com o símbolo \emph{⋆}. Isso indica que
o material dessa subseção pode ser ignorada sem consequências drásticas.

Licença

Este livro está licenciado sob Creative Commons Attribution-ShareAlike
4.0 (CC BY- SA 4.0) licence. Isso significa que você está convidado a
compartilhar o conteúdo desse livro,desde que você dê crédito ao autor e
que qualquer cópia ou derivados desde livro sejam liberados sob a mesma
licença.

A licenca pode ser lida em sua totalidade no final do livro ou seguindo
o URL:
\end{quote}

x \emph{Prefácio} Comentários e correções

\begin{quote}
Qualquer feedback, seja de alunos, professores assistentes, instrutores
ou quaisquer outros leitores, será muito apreciado. Particularmente
úteis são correções de erros tipográficos, sugestões de formas
alternativas de descrever conceitos ou provar teoremas e solicitações de
novos conteúdos. (e.g. se você conhecer um bom exemplo que possa
ilustrar um conceito, ou se tiver um conceito relevante que você
desejaria ver nesse livro).

Tal feedback pode ser mandado para o autor e adaptador por email (clive@
infinitedescent.xyz e jonas.agx@gmail.com, respectivamente).
\end{quote}

x

\begin{quote}
\textbf{Reconhecimentos}
\end{quote}

Quando reflito sobre o tempo que passei escrevendo este livro, fico
impressionado com o número de pessoas que tiveram algum tipo de
influência em seu conteúdo.

Este livro nunca teria existido se não fosse pelo curso 38-801 de Chad
Hershock\emph{Ensinos Baseados em Evidências ciêntificas},No qual fiz no
outono de 2014 como estudante de pós-graduação na Carnegie Mellon
University. Seu curso influenciou fortemente minha abordagem de ensino
e, em primeiro lugar, motivou-me a escrever este livro. Muitas das
decisões pedagógicas que tomei ao escrever este livro foram informadas
por pesquisas às quais fui exposto quando era aluno da turma de Chad.

O lendário professor da Carnegie Mellon, John Mackey, tem usado este
livro (em vários formatos) como notas de curso para 21-128
\emph{Conceitos Matemáticos e Provas} e 15-151 \emph{Fudamentos
Matemáticos da Ciência da Computação} Desde o outono de 2016. Sua
influência pode ser sentida ao longo de todo o livro: graças às
discussões com John, muitas provas foram reformuladas, seções
reestruturadas e explicações melhoradas. Como resultado, há alguma
sobreposição entre os exercícios deste livro e as questões nas folhas de
problemas. Estou extremamente grato por seu apoio contínuo.

Steve Awodey, que foi meu orientador de tese de doutorado, tem sido uma
fonte de inspiração para mim há muito tempo. Muitas das escolhas que fiz
ao escolher como apresentar o material deste livro baseiam-se no meu
desejo de fazer matemática. \emph{O caminho certo}---foi esse desejo que
me levou a estudar a teoria das categorias e, por fim, a me tornar aluno
de doutorado de Steve. Aprendi muito com ele e apreciei muito sua
paciência e flexibilidade em ajudar a direcionar minha pesquisa, apesar
de minha agenda lotada de ensino e de meus interesses extracurriculares
(como escrever este livro).

Talvez sem o conhecimento deles, muitas conversas esclarecedoras com as
seguintes pessoas ajudaram a moldar o material deste livro de uma forma
ou de outra: Jeremy Avigad, Deb Brandon, Santiago Cañez, Heather Dwyer,
Thomas Forster, Will Gunther, Kate Hamilton, Jessica Harrell, Bob
Harper, Brian Kell, Marsha Lovett, Ben Millwood,

xi

\emph{Index of LATEX commands}

\emph{Reconhecimentos}

\begin{quote}
David Offner, Ruth Poproski, Emily Riehl, Hilary Schuldt, Gareth Taylor,
Katie Walsh, Emily Weiss e Andy Zucker.

Uma rede \emph{Stack Exchange} influenciou o desenvolvimento deste livro
de duas maneiras importantes. Primeiro, sou um membro ativo do
\emph{Mathematics Stack Exchange} desde o início de 2012 e aprendi muito
sobre como explicar conceitos matemáticos de maneira eficaz;
ocasionalmente, uma pergunta sobre Mathematics Stack Exchange me inspira
a adicionar um novo exemplo ou

exercício ao livro. Em segundo lugar, tenho feito uso frequente do
\emph{LATEX Stack Exchange} para implementar alguns dos aspectos

mais técnicos de escrever um livro usando o LATEX .

O Departamento de Ciências Matemáticas da Carnegie Mellon University
apoiou-me academicamente, profissionalmente e financeiramente ao longo
do meu doutoramento e me apresentou mais oportunidades do que eu poderia
esperar para me desenvolver como professor. Este apoio é agora
continuado pelo Departamento de Matemática da Northwestern University,
onde trabalho atualmente como professor.

Gostaria também de agradecer a todos nos centros de ensino da Carnegie
Mellon e da Northwestern, no Eberly Center e no Searle Center,
respectivamente. Através de vários workshops, programas e bolsas em
ambos os centros de ensino, aprendi muito sobre como as pessoas aprendem
e transformei-me como professor. Sua abordagem da ciência do ensino e da
aprendizagem, centrada no aluno e baseada em evidências, está subjacente
a tudo o que faço como professor, inclusive ao escrever este livro ---
sua influência não pode ser subestimada.

Finalmente, e mais importante, sou grato aos mais de 1.000 alunos que já
usaram este livro para aprender matemática. Cada vez que um aluno entra
em contato comigo para tirar uma dúvida ou apontar um erro, o livro fica
melhor; isso se reflete nas dezenas de erros tipográficos que foram
corrigidos como consequência.
\end{quote}

Clive Newstead

\begin{longtable}[]{@{}
  >{\raggedright\arraybackslash}p{(\columnwidth - 4\tabcolsep) * \real{0.3333}}
  >{\raggedright\arraybackslash}p{(\columnwidth - 4\tabcolsep) * \real{0.3333}}
  >{\raggedright\arraybackslash}p{(\columnwidth - 4\tabcolsep) * \real{0.3333}}@{}}
\toprule()
\begin{minipage}[b]{\linewidth}\raggedright
Janeiro
\end{minipage} & \begin{minipage}[b]{\linewidth}\raggedright
de
\end{minipage} & \begin{minipage}[b]{\linewidth}\raggedright
\begin{quote}
2020
\end{quote}
\end{minipage} \\
\midrule()
\endhead
\bottomrule()
\end{longtable}

Evanston, Illinois

xii

\end{document}
