% Options for packages loaded elsewhere
\PassOptionsToPackage{unicode}{hyperref}
\PassOptionsToPackage{hyphens}{url}
%
\documentclass[
]{article}
\usepackage{amsmath,amssymb}
\usepackage{lmodern}
\usepackage{iftex}
\ifPDFTeX
  \usepackage[T1]{fontenc}
  \usepackage[utf8]{inputenc}
  \usepackage{textcomp} % provide euro and other symbols
\else % if luatex or xetex
  \usepackage{unicode-math}
  \defaultfontfeatures{Scale=MatchLowercase}
  \defaultfontfeatures[\rmfamily]{Ligatures=TeX,Scale=1}
\fi
% Use upquote if available, for straight quotes in verbatim environments
\IfFileExists{upquote.sty}{\usepackage{upquote}}{}
\IfFileExists{microtype.sty}{% use microtype if available
  \usepackage[]{microtype}
  \UseMicrotypeSet[protrusion]{basicmath} % disable protrusion for tt fonts
}{}
\makeatletter
\@ifundefined{KOMAClassName}{% if non-KOMA class
  \IfFileExists{parskip.sty}{%
    \usepackage{parskip}
  }{% else
    \setlength{\parindent}{0pt}
    \setlength{\parskip}{6pt plus 2pt minus 1pt}}
}{% if KOMA class
  \KOMAoptions{parskip=half}}
\makeatother
\usepackage{xcolor}
\usepackage{longtable,booktabs,array}
\usepackage{multirow}
\usepackage{calc} % for calculating minipage widths
% Correct order of tables after \paragraph or \subparagraph
\usepackage{etoolbox}
\makeatletter
\patchcmd\longtable{\par}{\if@noskipsec\mbox{}\fi\par}{}{}
\makeatother
% Allow footnotes in longtable head/foot
\IfFileExists{footnotehyper.sty}{\usepackage{footnotehyper}}{\usepackage{footnote}}
\makesavenoteenv{longtable}
\usepackage{graphicx}
\makeatletter
\def\maxwidth{\ifdim\Gin@nat@width>\linewidth\linewidth\else\Gin@nat@width\fi}
\def\maxheight{\ifdim\Gin@nat@height>\textheight\textheight\else\Gin@nat@height\fi}
\makeatother
% Scale images if necessary, so that they will not overflow the page
% margins by default, and it is still possible to overwrite the defaults
% using explicit options in \includegraphics[width, height, ...]{}
\setkeys{Gin}{width=\maxwidth,height=\maxheight,keepaspectratio}
% Set default figure placement to htbp
\makeatletter
\def\fps@figure{htbp}
\makeatother
\setlength{\emergencystretch}{3em} % prevent overfull lines
\providecommand{\tightlist}{%
  \setlength{\itemsep}{0pt}\setlength{\parskip}{0pt}}
\setcounter{secnumdepth}{-\maxdimen} % remove section numbering
\ifLuaTeX
  \usepackage{selnolig}  % disable illegal ligatures
\fi
\IfFileExists{bookmark.sty}{\usepackage{bookmark}}{\usepackage{hyperref}}
\IfFileExists{xurl.sty}{\usepackage{xurl}}{} % add URL line breaks if available
\urlstyle{same} % disable monospaced font for URLs
\hypersetup{
  hidelinks,
  pdfcreator={LaTeX via pandoc}}

\author{}
\date{}

\begin{document}

\begin{quote}
2 \emph{Index of LATEX commands}
\end{quote}

Chapter 1

\textbf{Estrutura Lógica}

\begin{quote}
O objetivo desse capítulo é desenvolver uma maneira metódica de dividir
uma preposição em componentes menores e ver como eles se encaixam---isso
é chamado \emph{a estrutura lógica} de uma preposição. A \emph{estrutura
lógica} de uma preposição é muito informática: nos diz o que precisamos
fazer para prova-la, o que precisamos descrever para comunicar a nossa
prova e como explorar as consequências da preposição depois da mesma ter
sido provada.
\end{quote}

Estrutura lógica de uma

\begin{longtable}[]{@{}
  >{\raggedright\arraybackslash}p{(\columnwidth - 4\tabcolsep) * \real{0.3333}}
  >{\raggedright\arraybackslash}p{(\columnwidth - 4\tabcolsep) * \real{0.3333}}
  >{\raggedright\arraybackslash}p{(\columnwidth - 4\tabcolsep) * \real{0.3333}}@{}}
\toprule()
\multirow{2}{*}{\begin{minipage}[b]{\linewidth}\raggedright
\includegraphics[width=0.88889in,height=0.40278in]{vertopal_708b3db82b3347f88495d9245cd66f5b/media/image1.png}
\end{minipage}} & \begin{minipage}[b]{\linewidth}\raggedright
\begin{quote}
preposição
\end{quote}
\end{minipage} &
\multirow{2}{*}{\begin{minipage}[b]{\linewidth}\raggedright
\begin{quote}
\includegraphics[width=0.81944in,height=0.40278in]{vertopal_708b3db82b3347f88495d9245cd66f5b/media/image2.png}
\end{quote}
\end{minipage}} \\
& \begin{minipage}[b]{\linewidth}\raggedright
\includegraphics[width=\textwidth,height=0.38889in]{vertopal_708b3db82b3347f88495d9245cd66f5b/media/image3.png}
\end{minipage} \\
\midrule()
\endhead
\bottomrule()
\end{longtable}

\begin{quote}
estratégias para provar estrutura e redação da consequências da the
propositi a preposição prova proposição

Sections 1.1 e 1.2 são dedicadas ao desenvolver um sistema de
\emph{lógicasimbólica} para raciocinar sobre preposições.Seremos capazes
de representar uma preposição utilizando uma série de variáveis e
símbolos, e esta expressão orientará como podemos provar a preposição e
explorar as suas consequências. Na Sections 1.3

preview\\
preview
\end{quote}

2

desenvolveremos técnicas para manipular essas expressões lógicas
algebricamente -isso, por sua vez, produzirá novas provas técnicas
(algumas têm nomes sofisticados como ``prova por contraposição''),mas
outras não). Explorar como a estrutura lógica de uma proposição informa
a estrutura e o texto de sua prova é o conteúdo do Apêndice A.2.

3

\end{document}
