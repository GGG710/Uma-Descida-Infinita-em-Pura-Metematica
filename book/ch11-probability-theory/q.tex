% !TeX root = ../../book.tex

\subsection*{True--False questions}

\tfquestiontext{cqProbabilityTFBegin}{cqProbabilityTFEnd}

\begin{chapex} % True
\label{cqProbabilityTFBegin}
In a probability space $(\Omega, \mathbb{P})$, we have $\mathbb{P}(A \mid \Omega) = \mathbb{P}(A)$ for all events $A$.
\end{chapex}

\begin{chapex} % True
The function $\mathbb{P} : \mathcal{P}(\mathbb{N}) \to [0,1]$ defined by $\mathbb{P}(A) = \sum_{n \in A} 2^{-n}$ is a probability measure on $\mathbb{N}$.
\end{chapex}

\begin{chapex} % False
The function $\mathbb{P} : \mathcal{P}(\mathbb{Z}) \to [0,1]$ defined by $\mathbb{P}(A) = \sum_{n \in A} 2^{-n}$ is a probability measure on $\mathbb{Z}$.
\end{chapex}

\begin{chapex} % False
The function $\mathbb{P} : \mathcal{P}(\mathbb{N}) \to [0,1]$, defined by letting $\mathbb{P}(\varnothing) = \varnothing$ and $\mathbb{P}(A) = 2^{-\mathrm{min}(A)}$ for all $A \ne \varnothing$, is a probability measure on $\mathbb{N}$.
\end{chapex}

\begin{chapex} % True
\label{cqProbabilityTFEnd}
A random variable can have infinite expectation.
\end{chapex}

\subsection*{Always--Sometimes--Never questions}

\asnquestiontext{cqProbabilityASNBegin}{cqProbabilityASNEnd}

\begin{chapex} % Sometimes
\label{cqProbabilityASNBegin}
Let $(\Omega, \mathbb{P})$ be a probability space. The only event whose probability is zero is the empty event $\varnothing \subseteq \Omega$.
\end{chapex}

\begin{chapex} % Never
Let $(\Omega, \mathbb{P})$ be a probability space and let $A$ be an event. Then $\mathbb{P}(A)^2 - \mathbb{P}(A) = 2$.
\end{chapex}

\begin{chapex} % Sometimes
Let $(\Omega, \mathbb{P})$ be a probability space and let $A$ and $B$ be events with $\mathbb{P}(B)>0$. Then $\mathbb{P}(A \mid B) \ge \mathbb{P}(A)$.
\end{chapex}

\begin{chapex} % Always
Let $(\Omega, \mathbb{P})$ be a probability space and let $A$ and $B$ be mutually exclusive events with $\mathbb{P}(B) > 0$. Then $\mathbb{P}(A \mid B) = 0$.
\end{chapex}

\begin{chapex} % Sometimes
\label{cqProbabilityASNEnd}
Let $(\Omega, \mathbb{P})$ be a probability space and let $A$, $B$ and $C$ be events with $\mathbb{P}(B) > 0$ and $\mathbb{P}(C) > 0$. Then $\mathbb{P}(A \mid B) \mathbb{P}(B \mid C) = \mathbb{P}(A \mid C)$.
\end{chapex}